\documentclass[]{article}

\usepackage{amsfonts}
\usepackage{amsmath}
\usepackage{amssymb}
\usepackage{amsthm}
\usepackage{caption}
\usepackage{enumitem}
\usepackage{setspace}
\usepackage{import}
\usepackage{xifthen}
\usepackage{pdfpages}
\usepackage{transparent}
\usepackage{hyperref}
\usepackage{esint}

\theoremstyle{definition}
\newtheorem{defi}{Definition}
\newtheorem{eg}{Example}
\newtheorem{lemma}{Lemma}
\newtheorem{thm}{Theorem}
\newtheorem*{axiom}{Axiom}

\newcommand{\C}{\mathbb{C}}
\newcommand{\N}{\mathbb{N}}
\newcommand{\Q}{\mathbb{Q}}
\newcommand{\R}{\mathbb{R}}
\newcommand{\Z}{\mathbb{Z}}

\newcommand{\incfig}[1]{
    \def\svgwidth{\columnwidth}
    \import{./figures/}{#1.pdf_tex}
}

\let\emptyset\varnothing

\DeclareMathOperator{\vspan}{span}




\title{Set Theory}
\author{Karan Elangovan}

\begin{document}

\maketitle

\doublespacing
\tableofcontents

\section{ZFC Axioms}

\subsection{ZF Axioms}

Throughout these notes we will implicitly assume all the rules and axioms of logic and avoid scrutinising or considering them on the basis that they are so obvious that any axiomatic treatment would be of a purely formal and philisophical interest, and would lead to no non-trivial results that could be applied to more interesting problems.

We define a variable as a string of characters possibly with subscripts, the symbol $\emptyset$ as the empty set, and $\cup, \mathcal{P}$ as function symbols. We now define a term inductively as either a variable, a non function symbol or a function symbol applied to a term. In ZFC every term is a set (though there will be as a consequence of the proceeding axioms sets which cannot be expressed as terms due to their infinite nature)

We define the atomic formulas as $x = y$ and $x \in y$, where $x, y$ are terms. We then define a formula as either an atomic formula or a logical connective or quantifier applied to a formula. 

Below we have the Zermelo-Frankel axioms, which alongside the Axiom of Choice comprise the entirety of ZFC. We list each one as an intuitive interpretation alongside the formal statement.

\begin{axiom} [Extensionality]
	If $a$ and $b$ have the same elements, then they are equal
	\[
			\forall x (x \in a \iff x \in b) \implies a = b
	.\] 
\end{axiom}

\begin{axiom} [Empty Set]
	The empty set, $\emptyset$, has no elements.

	\[
	\forall x (x \not \in \emptyset)	
	.\] 
\end{axiom}

\begin{axiom} [Pairing]
		Given any sets $x, y$, the set ${x,y}$ exists.

		\[
				\forall x \forall y \exists z (t \in z \iff (t = x \lor t = y))
		.\] 
\end{axiom}

\begin{axiom} [Union]
	The union of a family of sets behaves as we would expect.

	\[
			\forall t (t \in \cup X \iff \exists y (t \in y \land y \in X))	
	.\] 
\end{axiom}

\begin{axiom} [Power Set]
	The power set of a set behaves as we would expect.

	\[
			\forall t (t \in \mathcal{P}(X) \iff \forall x (x \in t \implies x \in X))			
	.\] 
\end{axiom}

\begin{axiom} [Separation]
		Given a set $X$ and a formula $\psi(x)$	which does not contain $S$ we have the set ${x \in X | \psi(x)}$.

		 \[
				 \exists S \forall x (x \in S \iff (x \in X \land \psi(x)))
		.\] 
\end{axiom}

\begin{axiom} [Infinity]
		There is a set $Z$ which contains $\emptyset$ and if $x \in Z$ then $x \cup \{x\} \in Z$	

		\[
				\exists Z(\emptyset \in Z \land \forall x(x \in Z \implies x \cup \{x\} \in Z))
		.\] 
\end{axiom}

\begin{axiom} [Replacement]
	We may restrict the range of a function to an arbitrary set.
	Let $\psi(x,y)$ be a formula.

	\[
			\forall x \forall y \forall z ((\psi(x,y) \land \psi(x,z) \implies y = z) \implies \forall D \exists R \forall y (y \in R \iff \exists x (x \in D \land \psi(x,y))))
	.\] 
\end{axiom}

\begin{axiom} [Regularity]
	Every non-empty $x$ contains $y$ disjoint from $x$.

	\[
			\forall x (x \neq \emptyset \implies \exists y (y \in x \land \forall z \lnot (z \in x \land z \in y)))
	.\] 
\end{axiom}

The next two theorems actually don't use any of the ZF axioms and are instead just based on pure logical deduction.

\begin{thm} \label{thm:sameelmsamesets}
	Equal sets contain the same elements, i.e. the converse of the axiom of extensionality holds.

	\[
			\forall x \forall y (x = y \implies \forall z (z \in x \iff z \in y))
	.\] 
\end{thm}

\begin{proof}
	Consider arbitrary $z$. Then $z \in x \iff z \in y$ as we may substitute $x$ for $y$ as they are equal.
\end{proof}

\begin{thm} \label{thm:transsubsets}
	The subset relation is transitive.

	\[
			\forall x \forall y \forall z ((x \subset y \land y \subset z) \implies x \subset z)
	.\] 
\end{thm}

\begin{proof}
		If $x \subset y$ and $y \subset z$, then $\forall t ((t \in x \implies t \in y) \land (t \in y \implies t \in z))$. Hence $\forall t (t \in x \implies t \in z)$, and so $x \subset z$.
\end{proof}

From here, wanting to prove some less trivial results we have to start invoking the ZF axioms.

\begin{thm} \label{thm:mutalsubsets}
	If $x \subset y$ and $y \subset x$, then $x = y$
\end{thm}

\begin{proof}
		As $x \subset y$, $\forall t (t \in x \implies t \in y)$, and as $y \subset x$, $\forall t(t \in y \implies t \in x)$. So $\forall t (t \in x \iff t \in y)$, which implies $x = y$ by the axiom of extensionality.
\end{proof}

\begin{thm} \label{thm:singleton}
		For any $x$ the singleton set $\{x\}$ exists. 
		\[
				\forall x \exists y(t \in y \iff t = x)
		.\] 
\end{thm}

\begin{proof}
		By substituting $y = x$ in the pairing axiom we have $\forall x \exists z \forall t(t \in z \iff (t \in x \land t \in x))$, which show the existence of the desired singleton set.
\end{proof}

\begin{thm} \label{thm:intersection}
		For any $X$ the set with the expected properties of $\cap X$ exists.
\end{thm}

\begin{proof}
		By the union and separation axioms we have the existence of the set $Y = \{x \in \cup X | \forall t(t \in X \implies x \in t)\}$, which is exactly the set whose properties we desired.
\end{proof}

\begin{thm} \label{thm:noselfcontain}
	No set contains itself.
\end{thm}

\begin{proof}
		Assume the contrary. That for some $x$ we have $x \in x$. By ~\ref{thm:singleton}, we have the set $\{x\}$. By the axiom of regularity $\{x\}$ contains an element disjoint from itself. The only element of $\{x\}$ is $x$, hence $x$ and $\{x\}$ are disjoint. But $x \in x$ and $x \in \{x\}$, so they are not disjoint, a contradiction.
\end{proof}

\subsection{Primitive Constructs}

We may now use the ZF axioms to construct various primitive objects in a rigorous manner.

\begin{defi} [Ordered Pair]
		We define the ordered pair $(x, y)$ as $\{x, \{x, y\}\}$ (this definition makes sense by the axiom of pairing)	
\end{defi}

\begin{defi} [Cartesian Product]
		We define the cartesian product of $A$ and $B$ as $\{(a, b) | a \in A \land b \in B\}$, this is possible by applying the axiom of separation on a suitable set that contains all the desired pairs, such as $\mathcal{P}(A \cup \mathcal{P}(A \cup B))$
\end{defi}

\begin{defi} [Partial Order]
	A partial order, $B$, of $X$ is a subset of $X \times X$ satisfying:
	\begin{enumerate}
			\item $\forall x(x \in X \implies (x,x) \in B$ (reflexivity)
			\item $\forall x \forall y(((x,y) \in B \land (y,x) \in B) \implies x = y)$ (antisymmetry)
			\item $\forall x \forall y \forall z(((x,y) \in B \land (y,z) \in B) \implies (x,z) \in B)$ (transitivity)
	\end{enumerate}

	We call a set with a partial order a poset.

	We call $x \in X$ a maximal element of $X$ under the partial order iff $\forall y((y \in X \land (x,y) \in B) \implies x = y)$.
\end{defi}

\begin{defi} [Total Order]
		A total order is a special case of a partial order. A partial order $B \subset X\times X$ is a total order if it also satisfies connexity: $\forall x \forall y((x \in X \land y \in X) \implies ((x,y) \in B \lor (y,x) \in B)$

		A subset $S$ of a partial order in which the ordering induced by the original set makes $S$ a total order is a chain.

		We call a subset $T$ of a chain $S$ an initial segment iff when $u \leq v$ are in $S$ and $v \in T$, then $u \in T$
\end{defi}

\begin{defi} [Well-Ordering]
	A total ordering $\leq$ over $X$ is a well-ordering iff every non-emtpy subset of $X$ has a least element.
\end{defi}

\begin{defi} [Function]

		We define a function $f$ mapeing $A$ to $B$ as any subset of $A \times B$ satisfying:
		\begin{itemize}
		\item $\forall x(x \in A \implies \exists y ((x,y) \in f)$
		\item $\forall x \forall y \forall z((x, y) \in f \land (x, z) \in f) \implies y = z$	
		\end{itemize}	

		We further say that $f$ is injective iff $\forall x_1 \forall x_2 \forall y((x_1, y) \in f \land (x_2, y) \in f) \implies x_1 = x_2$ and surjective iff $\forall y (y \in B \implies \exists x (x \in A \land (x,y) \in f)$. And that  $f$ is bijective iff it is injective and surjective.
\end{defi}

\subsection{The Axiom of Choice}

The axiom of choice allows us to choose an element from each set of a family of non empty sets. Formally, if $T$ is a set of non empty sets, then the AC states there exists a "choice" function $f: T \to \cup T$ satisfying $\forall t(t \in T \implies f(t) \in t)$.

An extremely useful consequence of the AC is Zorn's Lemma. It is actually equivalent to the AC under the other ZF axioms, however we will only show that it follows from ZFC.

\begin{thm} [Zorn's Lemma] \label{thm:zorns-lemma}
	If $P$ is a poset such that every chain in $P$ is bounded above under the partial order relation, then $P$ must have a maximal element.
\end{thm}

\begin{proof}
		We will proceed by contradiction. Assume that $P$ satisfies the given hypotheses but has no maximal element. Consider an arbitrary chain $S$, by hypothesis $S$ is bounded above. If there are no upper bounds outside of $S$, then $S$ would have a maximal element (the element at the "top of the chain"), hence the set of upper bounds of $S$ not in $S$, $B(S)$ is non-empty.

		Let $\mathcal{C}$ be the set of chains of $P$. Then by the AC we have $\varphi: \mathcal{C} \to B(\mathcal{C})$ such that $\varphi(S) \in B(S)$, i.e.  $\varphi$ chooses an upper bound of $S$ not in $S$ (as we have shown such an upper bound always exists)

		Now consider a fixed $p \in P$. Let $C$ be the set of well-ordered chains, $S$, satisfying
		\begin{itemize}
				\item $p$ is the least element of $S$ 
				\item For any proper non-empty initial segment $T$ of $S$, the least element of $S - T$ is $\varphi(T)$
		\end{itemize}

	Intuitively $C$ is the set of well-ordered chains "starting" at $p$ and such that for any initial segment $\varphi$ chooses the "next" element as the upper bound, in the context of the chain $S$

	Say $S, S' \in C$. We claim that either $S$ or $S'$ is an initial segment of the other. Let $R$ be the union of all sets that are simultaeneously initial segments of $S$ and $S'$, so $R$ is itself a common initial segment which is not contained in any other inital segments. Say $R$ is not equal to either of $S$ or $S'$.  $p \in R$, hence we have that $\varphi(T)$ is the least element of both $S - R$ and $S' - R$. Hence $R \cup \{\varphi(R)\}$ is a common initial segment containing $R$, which cannot be. Hence $R = S$ or $R = S'$. Hence one of $S$ and $S'$ is an initial segment of the other.

	Now let $U = \cup C$. In light of the previous paragraph, we have that  $U$ is a well-ordered chain with least element $p$. If $T$ is a proper non-empty inital segment of $U$, then we may choose $u \in U - T$, which means $u \in S - T$ for some $S \in C$. Again by the previous paragraph we have that $T$ is a proper non-empty initial segment of $s$, so $\varphi(T)$ is the least element fo $S - T$ and hence also of $U - T$, hence $U \in C$. Hence $U \cup \{\varphi(U)\} \in C$, but $(U \cup \{\varphi(U)\} \not \subset U$, a contradiction.

	Hence $P$ has a maximal element.
\end{proof}

\subsection{Construction of the Naturals}

	The axiom of infinity guarantees us the existence of an "infinite" set, however this gives us no information as to what kind of infinite set this would be. But a little manipulation allows us to construct a set that will satisfy the Peano axioms of arithmetic.

	\begin{defi} [Successor]
		We write the successor of a set $x$ as $S(x)$ to denote $x \cup \{x\}$
	\end{defi}
	
	\begin{defi} [Inductive Set]
		We call a set inductive if it satisfies the conditions of the axiom of infinity, that is it contains $\emptyset$ and if it contains $x$ then it contains $S(x)$. 
	\end{defi}

	\begin{defi} [Natural Numbers]
			We define the natural numbers $\mathbb{N}$ as the intersection of all inductive subsets of $Z$ (where $Z$ is the "infinite" set given by the axiom of infinity)
	\end{defi}

	\begin{thm} [Peano Arithmetic] \label{thm:peano}
			The natural numbers and successor we have specified satisfy the axioms of Peano Arithmetic (with the additional definition of $0 = \emptyset$), namely:
			\begin{itemize}
					\item $0$ is a natural number	
					\item The successor is an injective function from $\mathbb{N} \to \mathbb{N}$ 
					\item $0$ has no preimage under the successor function
					\item Any inductive set of naturals is $\mathbb{N}$ itself.
			\end{itemize}
	\end{thm}

	\begin{proof}
			$0$ is a natural number as every inductive set must contain $\emptyset$ by definition.

			The successor is a function as by definition if $x$ is in an inductive set, then so is $S(x)$. Say $S(x) = S(y)$ for $x \neq y$, then we have $x \cup \{x\} = y \cup \{y\}$, so they both contain the same elements.  $x$ is an element of LHS, hence $x \in y \cup \{y\}$, which means $x \in y$ as $x \not\in \{y\}$ as $x \neq y$. By similar logic $y \in x$, however this contradicts the axiom of regularity. Hence $S(x) = S(y) \implies x = y$, so $S$ is injective.

			$S(x) = x \cup \{x\}$, hence $x \in S(x)$, so $S(x) \neq \emptyset$, so $0$ has no preimage under the successor function.

			If $W$ is an inductive subset of $\mathbb{N}$, then $\mathbb{N} \subset W$ as $\mathbb{N}$ is the intersection of all inductive subets of $Z$. Hence $W = \mathbb{N}$	
	\end{proof}

	From here we may proceed to inductively define addition, subtraction and multiplication. Then we may define the integers and rationals as a pair of a "sign" (any of 2 distinct sets) and a natural, and as pairs of integers with the second non-zero respectively. Reals may then be defined using Dedekind cuts and complexes as orderd pairs of reals. We omit these definitions as they are trivial and time-consuming.

\section{Ordinals and Cardinals}

\subsection{Cardinality}

We may rephrase our intuitive notions of the relative sizes in terms of injective and surjective functions so as to generalise to infinite sets.

We say $X \precsim Y$ if there is a injective function from X to Y, that $X \sim Y$ if there is a bijection between $X$ and $Y$, and $X \prec Y$ if $X \precsim Y$ and $X \not\sim Y$. Intuitively the relation $\precsim$ should be thought of as saying one set is "smaller" than the other.

\begin{thm} [Cantor's Theorem] \label{thm:cantor}
		For any $X$, $X \prec \mathcal{P}(X)$	
\end{thm}

\begin{proof}
The function $f: X \to \mathcal{P}(X)$ defined by $f(x) = \{x\}$ is an injection and hence $X \precsim Y$. Now say that $X \sim \mathcal{P}(X)$, then we have a bijection $f: X \to \mathcal{P}(X)$. By the axiom of separation, we have the existence of $Y = \{x \in X | x \not \in f(x)\}$. By surjectivity of  $f$ we have the existence of some $x \in X$ such that $f(x) = Y$, as  $Y \subset X$. If $x \in Y$ then $x \not\in Y$ and the converse holds by the construction of $Y$. This means that $\lnot (x \in Y \lor x \not\in Y)$, which is a contradiction, hence  $X \not\sim \mathcal{P}(X)$. Hence we have $X \prec \mathcal{P}(X)$.
\end{proof}

\begin{thm} [Cantor-Bernstein Theorem] \label{thm:cantor-bernstein}
	If $A \precsim B$ and $B \precsim A$, then $A \sim B$.
\end{thm}

\begin{proof}
	By definition, we have injection  $f: A \to B$ and $g: B \to A$. Consider the sequence of sets 
	\begin{align*}
			A_0 = A \\
			A_1 = g(B) \\
			A_2 = g \circ f(A)\\
			A_3 = g \circ f \circ g(B)
	\end{align*}
	A trivial induction argument shows that $A_{n+1} \subset A_n$. So we may define the sets $A'_n = A_n - A_{n+1}$ and $A'_{\omega} = \bigcap\{A_n\}$. It is trivially shown that these sets form a partition of  $A$. We now define $h: A \to B$ by 
	\[
				h(x) = \begin{cases}
						f(x) \quad &\text{if} \, x \in \cup\{A'_{\omega}, A'_0, A'_2, \ldots\} \\
						g^{-1}(x) \quad &\text{if} \, x \in \cup\{A'_1, A'_3, A'_5, \ldots\}
				\end{cases}
	.\] 

	As the $A'_n$ forms a partition of $A$, $h$ is well-defined on $A$. By considering a similar sequence starting with $B$, it is trivially shown that $h$ is bijective. Hence $A \sim B$
\end{proof}

\subsection{Ordinals}

\begin{defi} [Ordinal]
		An ordinal is a transitive set which is well-ordered by the relation $\leq_\in$
\end{defi}

\begin{thm} \label{thm:new-ordinals}
	Let $X$ be a set of ordinals and $\alpha$ and $\beta$ be ordinals, then we have:
	\begin{itemize}
			\item $\alpha \cup \{\alpha\}$ is an ordinal
			\item $\alpha \in \beta$, or $\alpha = \beta$, or $\beta \in \alpha$
			\item $\cup X$ is an ordinal
	\end{itemize}
\end{thm}

\begin{proof}
		First we show $\alpha' = \alpha \cup \{\alpha\}$ is an ordinal. Consider arbitrary $t \in x \in \alpha'$. Then $x \in \alpha \lor x = \alpha$. If $x = \alpha$, then $y \in \alpha$, hence $y \in \alpha'$. If  $x \in \alpha$, then $y \in \alpha$ by the transtiivity of $\alpha$, and so $y \in \alpha'$. Hence $\alpha'$ is transitive. The relation $\leq_\in$ is antisymmetric for any set by the axiom of regularity. Say $x \leq_\in y$ and $y \leq_\in z$, we consider only when all the inequalities are strict as the other cases are trivial. So we have $x \in y \in z \in \alpha'$. If $z \in \alpha$, then $x \in z$ holds by $\alpha$ being an ordinal. If $z = \alpha$, then by the transitivity of $\alpha$, $x \in \alpha$, so we again have $x \in z$,  hence the relation $\leq_\in$ is transitive over $\alpha'$. Now consider arbitrary  $x, y \in \alpha'$. If $x$ and $y$ are both in $\alpha$, then we have connexity by the ordinality of $\alpha$. If $x = \alpha \land y = \alpha$ then $x = y$ and if $x \in \alpha \land y = \alpha$, then $x \in y$, and vice-versa. Hence  $\leq_\in$ satisfies trichotomy over $\alpha'$. If $\alpha'$ is not well-ordered by $\leq_\in$, then we would have an infinite descending chain of inclusion, which contradicts regularity.

	Next we show the trichotomy relationship between any ordinals. Let $C = \alpha \cap \beta$. Assume for contradiction that $C \neq \alpha$ and $C \neq \beta$. Then $\alpha - C$ is non-empty and so by the well-ordering of $\alpha$, there is a least element $\gamma$. For any $x \in \gamma$, $x \in \alpha$ by transitivity, which means $x \in \beta$, as otherwise would contradict the minimality of $\gamma$. Hence $\gamma \subset \beta$, and $\gamma \subset \alpha$ by transitivity of $\alpha$. Say that $x \in C$ but $x \not\in \gamma$. Then as $x, \gamma \in \alpha$ by trichotomy we have $x = \gamma$ or $\gamma \in x$. If the former is true then $\gamma \in C \subset \beta$, which cannot be. If the latter is true then $\gamma \in x \in \beta$, which means $\gamma \in \beta$ by transitivity, which also cannot be. Hence there are no such $x$, and so $\gamma \subset \alpha \cap \beta$ and $\alpha \cap \beta \subset \gamma$, meaning $\gamma = C$, so $C \in \alpha$. By similar logic, we have $C \in \beta$, and so $C \in C$, which contradicts the axiom of regularity. Hence $C = \alpha$ or $C = \beta$, hence $\alpha \subset \beta$ or $\beta \subset \alpha$. Assume the former case WLOG. If $\alpha = \beta$ then we are done. So we assume that $\alpha \subsetneq \beta$. So we may choose the least element $m$ of $\beta - \alpha$. If $x \in m$, then by minimality we have $x \in alpha$. Conversely if $x \in alpha$ then if $x = m$ or $m \in x$ then by transitivity in the latter case we have that $m \in \alpha$, which cannot be. Hence by trichotomy we have $x \in m$. So we have $\alpha = m \in \beta$.

	Finally we show the ordinals are closed under unions. Consider arbitrary $x \in y \in \cup X$, by the axiom of union we have  $t \in X$ such that $x \in y \in t$ which implies $x \in t$ as $t$ is an ordinal. So we have $x \in \cup X$, hence $\cup X$ is transitive. $\leq_\in$ satisfies antisymmetry over any set by regularity. Say $x \leq_\in y \leq_\in z \in \cup X$, again by the axiom of union we have $t \in \cup X$ such that $x \leq_\in y \leq_\in z \in t$, and so we have $x \leq_\in z$ as $t$ is an ordinal. Consider arbitary $x, y \in \cup X$, then by the axiom of union we have $t_1, t_2 \in X$ such that $x \in t_1$ and $y \in t_2$. $t_1$ and $t_2$ are ordinals, so we have one is a subset of the other. If $t_1 \subset t_2$ then $x, y \in t_2$ and so they are comparable by connexity, and similarly for $t_2 \subset t_1$. Hence $\leq_\in$ is a total ordering over $\cup X$. The must now be a well ordering as otherwise would imply an infinite sequence of inclusions, which contradicts the axiom of regularity. Hence $\cup X$ is an ordinal.
 \end{proof}

 \begin{thm} \label{thm:ordinals-canonical-well-ordering}
	Every non-empty well-ordered set $X$ is order isomorphic to a unique ordinal. 
 \end{thm}

 \begin{proof}
		 First we show uniqueness holds. Say ordinals $\alpha \neq \beta$ are order isomorphic, then by trichotomy of ordinals we may assume by symmetry that $\alpha \in \beta$. We have an order isomorphism $f: \beta \to \alpha$. From here we may construct the infinitely descending chain $\alpha, f(\alpha), f^2(\alpha), \ldots$, which contradicts the axiom of regularity. Hence if $\alpha$ and $\beta$ are ordinals and are order isomorphic to one another, then $\alpha = \beta$.

		 Now we show existence. Define the initial segment $I_x = \{t \in X | t < x\}$ for any $x \in X$ and $W$ be the set of all $x \in X$ such that $I_x$ is order isomorphic to an ordinal. By the uniqueness we have already shown, we may call this ordinal $\gamma_x$ for any $x \in W$. Consider $x, y \in W$ with $x < y$. Let $h_x: I_x \to \gamma_x$ and $h_y: I_y \to \gamma_y$ be the corresponding order isomorphisms. We claim that $h_x$ and $h_y$ agree for all $t < x$ (or equivelently $t \in I_x$ ). Say this is not the case, then we have some $t < x$ such that $h_x(t) \neq h_y(t)$, we consider only the case when  $h_x(t) < h_y(t)$, as the other case uses similar logic. We have $t < h_x^{-1}h_y(t)$, which yields the infinite descending chain $t > (h_x^{-1}h_y)t > (h_x^{-1}h_y)^2t > \ldots$, which contradicts the well-ordering of $X$. Hence our claim that $h_x$ and $h_y$ agree on $I_x$ is indeed true.
		 
		 Let $A$ be the set of ordinals that correspond to each $I_x$ for $x \in W$, and $\alpha = \cup A$ (so \alpha is an ordinal). In light of the previous paragraph, we have $h = \bigcup_{x \in W} h_x$ is an order isomorphism from $W$ onto $\alpha$. Now say that $W \neq X$, then let $t$ be the least element of $X - W \neq \emptyset$. Consider the function $h': (W \cup \{t\}) \to S(\alpha)$ defined by extending $h$ with $h'(t) = \alpha$.  Trivially this function restricted to $I_t$ is an order isomorphism onto $S(\alpha)$, meaning $t \in W$, a contradiction. Hence $W = X$, so $h$ is actually an order isomorphism from $X$ to the ordinal $\alpha$, hence $X$ is order isomorphic to an ordinal, showing existence.
 \end{proof}

 Intuitively, the above theorem says that we may treat the ordinals as the canonical well-ordered sets, and in particular that every set that admits a well-ordering is equal in cardinality to an ordinal. The next result shows that under the axiom of choice, this actually encompasses all sets.

 \begin{thm} [Well-Ordering Theorem] \label{thm:well-ordering}
 	Every set, $X$, admits a well-ordered relation.
 \end{thm}

 \begin{proof}
		 By the axiom of choice we have the existence of a function $k: \mathcal{P}(X) - \{\emptyset\} \to X$, and so we may define for every proper subset  $t$ of $X$, $x_t = k(X - t)$. Intuitively $x_t$ chooses an element of $X$ not in $t$.

		 We say that if for some ordinal $\alpha$, $f: \alpha \to X$ is an injection such that for all  $\beta < \alpha$ we have $f(\beta) = x_{f[\beta]}$, where $f[M]$ denotes the range of $f$ over the set $M$ (this set exists by the axiom of replacement, as we know that $f$ has range $X$), then $f$ is a "$\gamma$ function".

		 The key motivation for our strange criteria is that this guarantees that any two $\gamma$ functions $f: \alpha_1 \to X$ and $g: \alpha_2 \to X$ cohere on the intersection of their domains. Say this is not the case, then as  $\beta_1$ and $\beta_2$ are ordinals, we may choose the least $\beta \in \beta_1 \cap \beta_2$ such that $f(\beta) \neq g(\beta)$, which implies $x_{f[\beta]} \neq x_{g[\beta]}$, however this cannot be as $f$ and $g$ must cohere for all ordinals below $\beta$ by minimality of $\beta$. Hence $f$ and $g$ cohere as we claimed.

		 This coherence property, alongside the class of ordinals being totally ordered by the $\subset$ relation means that for any given subset $Y \subset X$, if there is a $\gamma$ function with image $Y$, then this function is unique. This, alongside the power set axiom allows us to construct the set of all $\gamma$ functions, $\Gamma$.

		 The coherence property, alongside injectivity means that  $h = \cup \Gamma$ is an injective function from some ordinal, $\theta$, (the union of a set of ordinals is an ordinal) to $X$. It is also trivial that $h$ is actually a $\gamma$ function. Indeed we claim that $h$ is also a surjective. Say this is not the case, then $h$ is non-surjective, and so $h[\theta]$ is a proper subset of $X$, and hence $x = x_{h[\theta]}$ exists. We define $h': S(\theta) \to X$ as equal to $h$ on $\theta$ and $h'(\theta) = x$. This satisfies $h'(\beta) = x_{h'[\beta]}$ trivially, and is also injective as by the construction of $x_t$, $x$ does not belong to $h[\theta]$, hence  $h'$ is a $\gamma$ function, which means $h' \subset h$, which is not the case. Hence $h$ is a bijection from a set of ordinals onto $X$. The class of ordinals is well ordered under $\leq_\in$, and so we have $X$ is well-ordered by the ordering induced by $h$.
 \end{proof}

 Whilst we have the existence of a well-ordering for any set, this says nothing as to how pathological and convoluted this well-ordering may be. $\N$ is actually well-ordered by the usual order relation. However for a set such as $\R$, the well ordering has no reason to take any useful form. So in the general case the key application of the Well-Ordering theorem is that it shows the ordinals suffice to measure the cardinality of any set.

 \begin{thm}
 		For any set $X$, there is a unique least ordinal $\alpha$ of equal cardinality to $X$.
 \end{thm}

 \begin{proof}
		 By the Well-Ordering theorem we have that $X$ may be well-ordered. And so by ~\ref{thm:ordinals-canonical-well-ordering} $X \sim \alpha$, for some ordinal $\alpha$. Let $A = \{\beta \leq \alpha | \beta \sim X\}$, then as $\alpha \in A$, so $A \neq \emptyset$, and the class of ordinals are well-ordered by inclusion, hence we may choose the least element $\beta$ of $A$. Say $\gamma \sim X$, then if $\gamma \leq \alpha$, then $\gamma \in A$ and so $\beta \leq \gamma$ by the minimality of $\beta$, otherwise $\gamma > \alpha \geq \beta$, yielding $\gamma \geq \beta$. Hence we have that $\beta$ is minimal amongst all ordinals equinumerous to $X$. 

		 If $\beta_1$ and $\beta_2$ are least equinumerous ordinals then $\beta_1 \leq \beta_2$ and $\beta_2 \leq \beta_1$, which implies $\beta_1 = \beta_2$, hence the ordinal is unique.
 \end{proof}

 In light of this result we may now make the following definitions.

 \begin{defi} [Cardinality]
 		The cardinality $|X|$ of $X$ is the unique least ordinal equinumerous to $X$.
 \end{defi}

 \begin{defi} [Cardinal]
 		A cardinal is a set $X$ which is its own cardinality.	
 \end{defi}

 Whilst we have spoken of sets having the same cardinality, now we actually have a definition of cardinality that makes sense for an induvidual set. Whilst the ordinals suffice to measure the cardinality of any set, they are not unique in this respect (except for the finite case), intuitively speaking as they also embody the structure of order (whilst they are not unique up to bijections, they are unique up to an order isomorphism), which is superflous when only considering cardinality. 
		
\end{document}

