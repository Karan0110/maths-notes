\documentclass[]{article}

\usepackage{amsfonts}
\usepackage{amsmath}
\usepackage{amssymb}
\usepackage{amsthm}
\usepackage{caption}
\usepackage{enumitem}
\usepackage{setspace}
\usepackage{import}
\usepackage{xifthen}
\usepackage{pdfpages}
\usepackage{transparent}
\usepackage{hyperref}
\usepackage{esint}

\theoremstyle{definition}
\newtheorem{defi}{Definition}
\newtheorem{eg}{Example}
\newtheorem{lemma}{Lemma}
\newtheorem{thm}{Theorem}
\newtheorem*{axiom}{Axiom}

\newcommand{\C}{\mathbb{C}}
\newcommand{\N}{\mathbb{N}}
\newcommand{\Q}{\mathbb{Q}}
\newcommand{\R}{\mathbb{R}}
\newcommand{\Z}{\mathbb{Z}}

\newcommand{\incfig}[1]{
    \def\svgwidth{\columnwidth}
    \import{./figures/}{#1.pdf_tex}
}

\let\emptyset\varnothing

\DeclareMathOperator{\vspan}{span}



\title{Group Theory}
\author{Karan Elangovan}

\begin{document}

\maketitle

\doublespacing
\tableofcontents

\section{Groups}

\subsection{Group Axioms}

\begin{defi} [Group]
		A group is an ordered pair $(G, \circ)$, where $\circ: G^2 \to G$ satisfying
		\begin{enumerate}
				\item (Associativity) The $\circ$ operation is associative.
				\item (Identity) There exists an element $1 \in G$ such that $1 \circ x = x \circ 1 = x$ for all $x \in G$.
				\item (Inverses) For every element $x$ there exists an element $x^{-1}$ such that $x \circ x^{-1} = x^{-1} \circ x = 1$.
		\end{enumerate}
\end{defi}

From here on we will omit the $\circ$ symbol and just denote it by adjacency, and when the group operation is clear from context, we will refer to a group by its set.

\begin{defi} [Subgroup]
	Given a subset $H \subset G$, we say it forms a subgroup if $H$ is a group under the induced group operation from $G$.
\end{defi}

\subsection{Cosets}

Given subsets $A, B \subset G$, we may define their product as
\begin{align*}
		AB = \{ab : a \in A, b \in B\}.
\end{align*}
That is, the set of all possible products of elements from $A$ and $B$. We see that from the associativity of element multiplication that multiplication of subsets is also associative.

In the case that $A$ or $B$ is a singleton, we just write $xB$ and $Ax$ for $\{x\}B$ and $A\{x\}$ respectively. The case that one of $A, B$ is a subgroup and the other a singleton is of particular interest, leading us to the following definition:

\begin{defi} [Cosets]
	Let $H$ be a subgroup of $G$. Then the left cosets of $H$ in $G$, $G / H$ are the subsets of the form $xH$ for all $x \in G$. We define the right cosets, $H \backslash G$ similarly.	
\end{defi}

We have the following results about left and right cosets.

\begin{thm}
		Let $H$ be a subgroup of $G$, and $x, y \in G$. Then all of the following hold
		\begin{enumerate}
				\item $Hx = Hy$ precisely when $xy^{-1} \in H$. 
				\item $|Hx| = |Hy|$.
				\item $|Hx| = |xH|$.
				\item $Hx = Hy$ or $Hx \cap Hy = \emptyset$.
				\item $|G / H| = |H \backlash G|$
		\end{enumerate}
\end{thm}

\begin{proof}
		First assume $Hx = Hy$, then right multiplying by $\{y^{-1}\}$ yields $H(xy^{-1}) = H$. As $H$ is a subgroup, $1 \in H$, so $1(xy^{-1}) = xy^{-1} \in H$. Now assume $xy^{-1} \in H$, then as subgroups are closed under multiplication, we have $Hxy^{-1} = H$, so multiplying by $\{y\}$ yields $Hx = Hy$. Hence we have (1).

		Define $f: Hx \to Hy$ by $f(g) = gx^{-1}y$. This has inverse $f^{-1}(g) = gy^{-1}x$, so $f$ is bijection, so $|Hx| = |Hy|$.	A similar bijection may be defined for (3).

		Say that $Hx \cap Hy \neq \emptyset$, so we may choose $g \in Hx \cap Hy$, so we have $h_1, h_2 \in H$ such that $g = h_1x = h_2y$, yielding $xy^{-1} = h_1^{-1}h_2 \in H$, meaning $Hx = Hy$.
\end{proof}

Of course by symmetry similar results hold for right cosets.

We now show that there are the same number of left and right cosets.

\begin{thm}
		Let $H$ be a subgroup of $G$. Then $|G / H| = |H \backslash G|$.
\end{thm}

\begin{proof}
	We define the map $f: G / H \to H \backslash G$	by
	\begin{align*}
			f(gH) = Hg.
	\end{align*}
	We first need to show that $f$ is well-defined. Say $g_1H = g_2H$, then $g_1g_2^{-1} \in H$, so $Hg_1 = Hg_2$. Similar logic shows that $f$ is an injection. We also have $f$ is trivially a surjection. So $f$ is a bijection between the left and right cosets. Hence we have $|G / H| = |H \backslash G|$.
\end{proof}

In light of the previous result, we make the following definition:

\begin{defi} [Index of a Subgroup]
		Let $H$ be a subgroup of $G$. We define the index of $H$ in $G$, $[G:H]$ as the common cardinality of the left cosets and the right cosets of $H$ in $G$, $|G / H| = |H \backslash G|$.
\end{defi}

In light of our work, the next theorem now follows trivially.

\begin{thm} [Lagrange's Theorem]
	Let $G$ be a subgroup of $H$, then we have
	\begin{align*}
			|G| = [G:H]|H|.
	\end{align*}
\end{thm}

\begin{proof}
		The left cosets of $H$ in $G$ form a partition of $G$, with each equivalency class of cardinality $|H|$. Hence we have $|G| = |H|[G:H]$.
\end{proof}

\begin{thm}
	Let $K \leq L \leq G$, then we have 
	\begin{align*}
			[G:L][L:K] = [G:K]
	\end{align*}
\end{thm}

\begin{proof}
		Let $A$ and $B$ be left transversals of $G / L$ and $L / K$, respectively. 

		First we note that if $a, a' \in A$ and $b, b' \in B$ are such that $abK = a'b'K$, then we have
		\begin{align*}
				abKL = a'b'KL \\
				abL = a'b'L \\
				aL = a'L \\
				a = a' \\
				b = b'.
		\end{align*}
		Hence $abK = a'b'K$ precisely when $a = a'$ and $b = b'$.

		Define the mapping $f: A \times B \to AB$ by 
		\begin{align*}
				f(a,b) = ab.
		\end{align*}
		This is trivially surjective and is injective by our initial observation. Hence we have
		\begin{align*}
				|AB| = |A||B| \\
				= [G:L][L:K].
		\end{align*}

		Define the mapping $g: AB \to G / K$ by 
		\begin{align*}
				g(x) = xK.
		\end{align*}
		By the initial observation this is injective. We see that
		\begin{align*}
				\cup g(AB) = ABK \\
				= AL \\
				= G.
		\end{align*}
		So we have that $g$ is surjective. Hence we have
		\begin{align*}
				[G : K] = |AB| \\
				= [G:L][L:K].
		\end{align*}
\end{proof}

The above result is a generalisation of Lagrange's Theorem, which follows by letting $K$ be the trivial subgroup.

We now derive an upper bound for the index of finite intersections of subgroups.

\begin{thm}
		Let $H_1, H_2, \ldots, H_m \leq G$. Then we have
		\begin{align*}
				[G : \cap_{i=1}^n H_i] \leq \prod_{i = 1}^n [G : H_i].
		\end{align*}
\end{thm}

\begin{proof}
		Consider $H, K \leq G$. Define the mapping $f: K / (H \cap K) \to G / H$ by 
		\begin{align*}
				f(k (H\cap K)) = kH.
		\end{align*}
		This is well-defined as $H\cap K \leq H$. Now we show that $f$ is injective:
		\begin{align*}
				f(k_1 (H \cap K)) = f(k_2 (H \cap K)) \\
				k_1 H = k_2 H \\
				k_1 k_2^{-1} \in H \cap K \\
				k_1 (H \cap K) = k_2 (H \cap K).
		\end{align*}
		So we have that 
		\begin{align*}
				[K : H \cap K] \leq [G : H] \\
				[G : K][K : H \cap K] \leq [G : K][G : H] \\
				[G : H \cap K] \leq [G : K][G : H].
		\end{align*}

		Hence the result for an arbitrary finite intersection follows from a trivial induction.
\end{proof}

In particular this means that a finite intersection of subgroups of finite index has a finite index itself. 

Another natural product to consider is that between two subgroups, $H$ and $K$. Of course in general this is not a subgroup itself, however we may derive a formula for the cardinality of this product when $H$ and $K$ are finite.

\begin{thm}
		Let $H$ and $K$ be finite subgroups of $G$. Then we have
		\begin{align*}
				|HK| = \frac{|H| |K|}{|H \cap K|}.
		\end{align*}
\end{thm}

\begin{proof}
		We note that 
		\begin{align*}
				HK = \cup \{Hk\}_{k\in K}
		\end{align*}

		Now say that $Hk_1 = Hk_2$ for $k_1, k_2 \in K$, then we have that $k_1k_2^{-1} \in H \cap K$, or equivalently $(H \cap K) k_1 = (H \cap K) k_2$. The converse follows trivially. So we have a bijective correspondence between $\{Hk\}_{k\in K}$ and the right cosets of  $H \cap K$ in $K$.

		Hence we have
		\begin{align*}
			|HK| = |H|[K : H\cap K] \\
			= \frac{|H| |K|}{|H \cap K|}.
		\end{align*}
\end{proof}

\subsection{Subgroup Generation}

An arbitrary subset $X \subset G$ is not necessarily a subgroup under the induced operation. So we aim to define the notion of the "smallest" subgroup that contains $X$.

\begin{defi} [Generated Subgroup]
	Let $X \subset G$. Then we define the subgroup generated by $X$ as the intersection of all subgroups which contain $X$.
\end{defi}

This does indeed define a subgroup, as $G$ itself contains $X$, so the intersection is non-empty. This definition also trivially means that $\langle X \rangle $ is minimal amongst subgroups containing $X$. 

An alternative way in which to think of $\langle X \rangle $ is as the set of all possible compositions of elements of  $X$ and their inverses, as of course this belongs to $\langle X \rangle $ and vice versa. To this end we make the following defintion:

\begin{defi} [Word]
		Let $X \subset G$. Then we define a word on $X$ as any finite sequence of elements of $X$ and their inverses.

		We define the value of a word to be the product of the terms in order.
\end{defi}

In essence, a word is a formalisation to speak of the actual terms in a product instead of merely the value of the product itself. In this way we see that the values of all words on $X$ are closed under inverses and products, and so we see that the set of all possible values of words is $\langle X \rangle $.

\end{document}

