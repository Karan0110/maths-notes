\documentclass[]{article}
 
\usepackage{amsfonts}
\usepackage{amsmath}
\usepackage{amssymb}
\usepackage{amsthm}
\usepackage{caption}
\usepackage{enumitem}
\usepackage{setspace}
\usepackage{import}
\usepackage{xifthen}
\usepackage{pdfpages}
\usepackage{transparent}
\usepackage{hyperref}
\usepackage{esint}

\theoremstyle{definition}
\newtheorem{defi}{Definition}
\newtheorem{eg}{Example}
\newtheorem{lemma}{Lemma}
\newtheorem{thm}{Theorem}
\newtheorem*{axiom}{Axiom}

\newcommand{\C}{\mathbb{C}}
\newcommand{\N}{\mathbb{N}}
\newcommand{\Q}{\mathbb{Q}}
\newcommand{\R}{\mathbb{R}}
\newcommand{\Z}{\mathbb{Z}}

\newcommand{\incfig}[1]{
    \def\svgwidth{\columnwidth}
    \import{./figures/}{#1.pdf_tex}
}

\let\emptyset\varnothing

\DeclareMathOperator{\vspan}{span}


 
\title{Differential Equations}
\author{Karan Elangovan}
 
\begin{document}
 
\maketitle

\doublespacing
\tableofcontents

\section{First Order Equations}

\subsection{Homogeneous Equations}

A homogenous differential equation is one of the form

\[
		M(x,y) dx + N(x,y) dy = 0
.\] 

Where $M$ and $N$ are homogenous functions of equal degree. A trivial rearrangement yields 

\[
		\frac{dy}{dx} = f(x,y)
.\] 

Where $f$ is a homogenous function of degree $0$. This means that $f(x,y) = f(tx, ty)$, and in particular that $f(x,y) = f(1, \frac{y}{x})$. Letting $z = \frac{y}{x}$, we have $zx = y$, and implicitly differentiating we have $xdz + zdx = dy$, which gives $\frac{dy}{dx} = z + x \frac{dz}{dx}$. Applying these results to the above equation yields

\[
		z + x \frac{dz}{dx} = f(1,z)
.\] 

This is a separable equation.


\begin{eg}
		Solve $(x+y) dx - (x-y) dy = 0$.

		We note that the $dx$ and $dy$ coefficients are homogenous of equal degree (degree 1), and so we may rearrange and apply the substitution $z = \frac{y}{x}$, to yield a separable equation.

		\begin{align*}
				x \frac{dz}{dx} + z = \frac{1+z}{1-z} \\
				\frac{1-z}{1+z^2} dz = \frac{1}{x} dx \\
				\log x = \arctan z - \log \sqrt{1+z^2} + C \\
				\arctan \frac{y}{x} = \log \sqrt{x^2 + y^2} + C
		\end{align*}
\end{eg}

\subsection{Exact Equations}

We say a differential equation is exact if it is of the form 

\[
		\frac{\partial f}{\partial x} dx + \frac{\partial f}{\partial y} dy = 0
.\] 

In this case the equation may be re-written as $df = 0$, meaning the solution is $f(x,y) = C$, for some constant $C$. However in general it may not be apparent from inspection that an equation of the form $M dx + N dy = 0$ is exact. It may be trivially shown that the equation is exact iff $\frac{\partial M}{\partial y} = \frac{\partial N}{\partial x}$.

\begin{eg}
		Solve $(\sin x \tan y + 1) dx - (\cos x \sec^2 y) dy = 0$.

	\begin{align*}
		\frac{\partial M}{\partial y} = \sin x \sec^2 y
		\frac{\partial N}{\partial x} = \sin x \sec^2 y
	\end{align*}

	Hence the equation is exact, so we have there is a function $f$ such that the LHS is $df$.

	\begin{align*}
			\frac{\partial f}{\partial x} = \sin x \tan y + 1 \\
			f = \int (\sin x \tan y + 1) dx \\
			= -\tan y \cos x + x + g(y)
	\end{align*}

	Differentiating with respect to $y$ 

	\begin{align*}
			-\cos x \sec^2 y = -\cos x \sec^2 y + g'(y)
			g'(y) = 0
			g(y) = C
	\end{align*}

	Hence we have

	\begin{align*}
			f = x - \tan y \cos x + C
	\end{align*}

	So the solution to the equation is 

	\begin{align*}
			\tan y \cos x - x = C
	\end{align*}

\end{eg}
 
\section{Miscellaneous}

\subsection{Motion of a Pendulum}

\begin{figure}[ht]
    \centering
    \incfig{pendulum}
    \caption{}
    \label{fig:pendulum}
\end{figure}

We model a pendulum as a light rod of length $l$ with one end fixed and to the other attatched a particle of mass $m$ that is released from rest at an angle $\alpha$ from the vertical.

We may obtain a differential equation for its motion in terms of its angle $\theta$ from the vertical by using that the sum of the particles kinetic and gravitational potential energy is constant.

\[
		\frac{1}{2} m (l\dot{\theta})^2 + mgl(1 - \cos{\theta}) = mgl(1 - \cos\alpha)
.\] 

Rearranging and considering the descent of the pendulum to its minimum point yields

\[
		dt = -\sqrt \frac{l}{2g} \frac{d\theta}{\sqrt{cos\theta - cos\alpha}}
.\] 

We now integrate from $t=0$ to the minimum point of the pendulum

\[
		\frac{T}{4} = \sqrt \frac{l}{2g} \int_{0}^\alpha \frac{d\theta}{\sqrt{\cos\theta - \cos\alpha}} 
.\] 

We now begin a series of manipulations with the aim of transforming this into a relatively simple expression of an elliptic integral of the first kind. Applying the double angle formula

\[
		\frac{T}{4} = \sqrt \frac{l}{2g} \int_0^\alpha \frac{d\theta}{\sqrt{2} \sqrt{\sin^2 \frac{\alpha}{2} - sin^2 \frac{\theta}{2}}}
.\] 

Let $k = \sin \frac{\alpha}{2}$. Now we make the substitution $\sin \frac{\theta}{2} = k\sin\phi$. So $\frac{1}{2} \cos \frac{\theta}{2} d\theta = k \cos\phi d\phi$, which written in terms of $d\theta$ yields $d\theta = \frac{2k\cos\phi}{\sqrt{1 - k^2sin^2\phi}}$. So we have the time period of the pendulum as 

\[
		T = 4 \sqrt \frac{l}{g} F\left(\sin \frac{\alpha}{2}, \frac{\pi}{2}\right)
.\]

where $F(k, \phi)$ denotes the elliptic integral of the first kind.
 
\end{document}
