\documentclass[]{article}

\usepackage{amsfonts}
\usepackage{amsmath}
\usepackage{amssymb}
\usepackage{amsthm}
\usepackage{caption}
\usepackage{enumitem}
\usepackage{setspace}
\usepackage{import}
\usepackage{xifthen}
\usepackage{pdfpages}
\usepackage{transparent}
\usepackage{hyperref}
\usepackage{esint}

\theoremstyle{definition}
\newtheorem{defi}{Definition}
\newtheorem{eg}{Example}
\newtheorem{lemma}{Lemma}
\newtheorem{thm}{Theorem}
\newtheorem*{axiom}{Axiom}

\newcommand{\C}{\mathbb{C}}
\newcommand{\N}{\mathbb{N}}
\newcommand{\Q}{\mathbb{Q}}
\newcommand{\R}{\mathbb{R}}
\newcommand{\Z}{\mathbb{Z}}

\newcommand{\incfig}[1]{
    \def\svgwidth{\columnwidth}
    \import{./figures/}{#1.pdf_tex}
}

\let\emptyset\varnothing

\DeclareMathOperator{\vspan}{span}




\title{Real Analysis}
\author{Karan Elangovan}

\begin{document}

\maketitle

\doublespacing
\tableofcontents

\section{Limits and Continuity}

\subsection{Limits}

The value of a function $f$ at $a$, in the abscence of any other information about $f$, gives absolutely no information about $f$ for values close to $a$. A behaviour that is of significant interest is when $f$ "approaches" a value (which is not necessarily $f a$) at $a$, in the specific sense that by considering a sufficiently small neighborhood of $a$, all the images of $f$ are arbitrarily close to $a$. We formalise this intuition in defining the limit of $f$ at $a$.

\begin{defi}[Limit]
		$l$ is the limit of $f$ at $a$, or symbolically $\lim_{x\to a} f(x) = l$ if and only if
		\[
				\forall \epsilon > 0 \exists \delta > 0: \forall 0 < |x - a| < \delta: |f(x) - l| < \epsilon
		.\] 
\end{defi}


\end{document}
