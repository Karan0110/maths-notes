\documentclass[]{article}

\usepackage{amsfonts}
\usepackage{amsmath}
\usepackage{amssymb}
\usepackage{amsthm}
\usepackage{caption}
\usepackage{enumitem}
\usepackage{setspace}
\usepackage{import}
\usepackage{xifthen}
\usepackage{pdfpages}
\usepackage{transparent}

\theoremstyle{definition}
\newtheorem{defi}{Definition}
\newtheorem{eg}{Example}
\newtheorem{lemma}{Lemma}
\newtheorem{thm}{Theorem}
\newtheorem*{axiom}{Axiom}

\newcommand{\C}{\mathbb{C}}
\newcommand{\N}{\mathbb{N}}
\newcommand{\Q}{\mathbb{Q}}
\newcommand{\R}{\mathbb{R}}
\newcommand{\Z}{\mathbb{Z}}

\newcommand{\incfig}[1]{
    \def\svgwidth{\columnwidth}
    \import{./figures/}{#1.pdf_tex}
}

\let\emptyset\varnothing



\title{Topology}
\author{Karan Elangovan}

\begin{document}

\maketitle

\doublespacing
\tableofcontents

\section{Topological Spaces}

\subsection{Topologies}

\begin{defi} [Topology]
		Given a set $X$, we say a collection of subsets of $X$, $\mathcal{T}$, is a topology if
		\begin{enumerate}
				\item $\emptyset, X \in \mathcal{T}$.
				\item $\mathcal{T}$ is closed under arbitrary unions
				\item $\mathcal{T}$ is closed under finite intersections.
		\end{enumerate}

		We call a set $X$ equipped with a topology $\mathcal{T}$ a topological space and the members of $\mathcal{T}$ open sets.
\end{defi}

\begin{defi} [Fineness]
		Given topologies $\mathcal{T}$ and $\mathcal{T}'$ on $X$, we say $\mathcal{T}$ is finer than $\mathcal{T}'$ if $\mathcal{T}' \subset \mathcal{T}$ and we define coarser similarly.

		We say two topologies are comparable if one is finer or coarser than the other.
\end{defi}

For an arbitrary set, $X$, we always have the discrete topology consisting of all subsets of $X$ and the indiscrete topology $\{\emptyset, X\}$. Hence the discrete topology is finer than every topology and the indiscrete topology is coarser than every other topology.

\subsection{Basis of a Topology}

Often there is no way to simply describe every possible open set of a topology, so we wish to describe it instead in terms of special open sets that in a sense "make up" the entire topology. We call this collection of "special" open sets a basis.

\begin{defi} [Basis]
	Let $X$ be a set. Then we say a collection, $\mathcal{B}$, of subsets of $X$ is a basis if
	\begin{enumerate}
			\item Every $x \in X$ is contained in a basis element.
			\item For any basis elements $B_1, B_2$, if $x \in B_1 \cap B_2$, then we have a basis element $B_3$ such that
					\begin{align*}
							x \in B_3 \subset B_1 \cap B_2.
					\end{align*}
	\end{enumerate}

	We then define the topology, $\mathcal{T}$, generated by $\mathcal{B}$ to be such that a set $U$ is open precisely when for every point $x \in U$ we have a basis element such that
	\begin{align*}
			x \in B \subset U.
	\end{align*}
\end{defi}

It is trivial to verify that the topology generated by a basis actually satisfies the axioms for a topology. Also as the term may suggest, the basis elements that generate a topology all belong to the topology. 

We may alternatively characterise basis sets in a more natural way

\begin{thm}
	Let $\mathcal{B}$ be a basis of $X$ that generates $\mathcal{T}$. Then we have
	\begin{enumerate}
			\item $\mathcal{T}$ is the intersection of all topologies on $X$ that contain $\mathcal{B}$.
			\item $\mathcal{T}$ is the set of all unions of collections of basis elements.
	\end{enumerate}
\end{thm}

\begin{proof}
	Let $ \mathcal{T}'$ be an arbitrary topology of $X$ that contains $\mathcal{B}$. Consider an arbitrary $U \in \mathcal{T}$. 

	We have for every $x \in U$ there exists a $B_x \in \mathcal{B} \subset \mathcal{T}'$ such that
	\begin{align*}
		x \in B_x \subset U.	
	\end{align*}
	Hence we have
	\begin{align*}
			U = \cup \{B_x\}_{x \in U}
	\end{align*}
	That is, $U$ is a union of sets open in $\mathcal{T}'$, so $U \in \mathcal{T}'$. Hence $\mathcal{T} \subset \mathcal{T}'$.

	$\mathcal{T}$  is a topology that contains $\mathcal{B}$ itself, so we have $\mathcal{T}$ is the intersection of all such topologies on $X$.
	
	The basis elements are open themselves, so any union of basis elements is open. For any $U \in \mathcal{T}$ we have for every $x \in U$ a basis element $B_x$ such that
	\begin{align*}
			x \in B_x \subset U.
	\end{align*}
	So we have
	\begin{align*}
			U = \cup \{B_x\}_{x \in U}
	\end{align*}
	Hence $U$ is union of a collection of basis elements. Hence $\mathcal{T}$ is the set of all unions of collections of basis elements.
\end{proof}

This means that the topology generated by a basis is the minimal (under the partial order of the subset relation) topology that contains the basis.

We now characterise the basis sets that generate a topology.

\begin{thm}
	Let $\mathcal{T}$ be a topology on $X$. 

	Then $\mathcal{B}$ is a basis that generates $\mathcal{T}$ if and only if every member of $\mathcal{B}$ is open and for every point $x$ in every open set $U$ we have a $B \in \mathcal{B}$ such that
	\begin{align*}
		x \in B \subset U.			
	\end{align*}
\end{thm}

\begin{proof}
	The forwards implication is by definition. 	

	So assume that $\mathcal{B}$ satisfies the latter hypothesis. 

	As $X$ is open, we have that for every $x \in X$ there is a basis element $B$ with $x \in B$. For any $B_1, B_2 \in \mathcal{B} \subset \mathcal{T}$, we have $B_1 \cap B_2$ is open, so we have a $B_3 \in \mathcal{B}$ such that
	\begin{align*}
			x \in B_3 \subset B_1 \cap B_2.
	\end{align*}
	Hence $\mathcal{B}$ is a basis.

	As $\mathcal{B} \subset \mathcal{T}$, the topology generated by $\mathcal{B}$ is coarser than $\mathcal{T}$. By definition every open set is in the topology generated by the basis. Hence $\mathcal{B}$ is a basis that generates $\mathcal{T}$.
\end{proof}

We now consider how to compare two topologies using bases that generate them.

\begin{thm}
		Let the basis $\mathcal{B}$ and $\mathcal{B}'$ generate the topologies $\mathcal{T}$ and $\mathcal{T}'$ on $X$. Then the following are equivalent
		\begin{enumerate}
				\item $\mathcal{T}'$ is finer than $\mathcal{T}$.
				\item For every $B \in \mathcal{B}$ and every $x \in B$ we have $B' \in \mathcal{B}'$ such that
						\begin{align*}
								x \in B' \subset B.
						\end{align*}
		\end{enumerate}
\end{thm}

\begin{proof}
		Assume that $\mathcal{T}'$ is finer than $\mathcal{T}$.	Then every $B \in \mathcal{B}$ is open is $\mathcal{T}'$, so for every $x \in B$ we have $B' \in \mathcal{B}'$ such that
		\begin{align*}
			x \in B' \subset B.	
		\end{align*}

		Now assume the latter hypothesis. Say $U$ is open in $\mathcal{T}$.  Then for every $x \in U$ we have a $B \in \mathcal{B}$ such that
		\begin{align*}
			x \in B \subset U.	
		\end{align*}
		We also have a $B' \in \mathcal{B}'$ such that
		\begin{align*}
				x \in B' \subset B.
		\end{align*}
		Hence we have
		\begin{align*}
				x \in B' \subset U.
		\end{align*}
		Hence $U$ is open in $\mathcal{T}'$.
\end{proof}

Considering the characterisation of the topology generated by a basis as the minimal topology containing the basis, we are motivated to generalise the notion of a basis.

\begin{defi} [Subbasis]
	Let $\mathcal{S}$ be a collection of subsets of $X$. Then $\mathcal{S}$ is a subbasis of $X$ if for every $x \in X$ we have a $S \in \mathcal{S}$ such that $x \in S$.


	We then define the topology generated by $\mathcal{S}$ to be the collection of all unions of finite intersections of $\mathcal{S}$.
\end{defi}

\subsection{Examples of Topologies}

We now define various useful ways to impose topologies on sets with some kind of structure.

\begin{defi} [Order Topology]
	Let $X$ be a totally ordered set. Then we define the order topology as the topology generated by the basis consisting of
	\begin{enumerate}
			\item All open intervals.
			\item All intervals of the form $(a,b_0]$ when there is a maximal element $b_0$.
			\item All intervals of the form $[a_0,b)$ when there is a minimal element $a_0$.
	\end{enumerate}
\end{defi}

\begin{defi} [Product Topology]
	Let $X$ and $Y$ be topological spaces. Then we define the product topology on $X \times Y$ as the topoology generated by the basis consisting of all products of open sets in $X$ and $Y$.		
\end{defi}

\begin{thm}
		Let $\mathcal{B}$ and $\mathcal{C}$ be bases for the topologies  on $X$ and $Y$, respectively. Then the products of basis elements of $X$ and $Y$ form a basis for $X \times Y$.
\end{thm}

\begin{proof}
		Consider an arbitrary basis element $U \times V$ of $X \times Y$. Then for every $(x,y) \in U \times V$ we have $B_x \in \mathcal{B}$ and $C_x \in \mathcal{C}$ such that
		\begin{align*}
				(x,y) \in B_x \times C_y \subset U \times V.
		\end{align*}
		Hence we have
		\begin{align*}
				U \times V = \cup \{B_x \times C_y\}_{x \in U, y \in V}.
		\end{align*}
		So the products of basis elements generate the product topology.
\end{proof}

\begin{defi} [Subspace Topology]
	Let $Y \subset X$. Then we define the subspace topology on $Y$ to be 
	\begin{align*}
			\mathcal{T}_Y = \{U \cap Y : U \in \mathcal{T}\}
	\end{align*}
\end{defi}

\end{document}
